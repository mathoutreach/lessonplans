\documentclass[12pt]{report}

\usepackage{graphicx}


\title{Symmetries of equilateral triangles}


\begin{document}

\thispagestyle{empty}

The group of symmetries of an equilateral triangle is size 6.  In
addition to the identity, there are two rotations, and three flips.
We can label the vertices to track how they move under the action of
this group.  While the vertices may move, the shape stays the same; this
is what we mean by symmetries.  They are:
\begin{figure*}
   \centering
   \begin{tabular}{|c|c|c|}
     \hline
     Nothing! & $1$ & \includegraphics[width=2cm]{id.pdf}\\
     \hline
     right & $r$ & \includegraphics[width=2cm]{r.pdf}\\
     \hline
     left & $\ell$ &\includegraphics[width=2cm]{l.pdf}\\
     \hline
     flip & $f$ & \includegraphics[width=2cm]{f.pdf}\\
     \hline
     flip right & $f_r$ &\includegraphics[width=2cm]{fr.pdf}\\
     \hline
     flip left & $f_\ell$ & \includegraphics[width=2cm]{fl.pdf}\\
     \hline
   \end{tabular}
\end{figure*}

If we were to perform on action and then another action, we would
still get the same shape, but the vertices get moved somewhere else.
Since there are only six possibilities for where they might end up,
each combination of symmetries has got to be one of the six symmetries
that we've already found!

For example, if we first rotate to the right ($r$), and then flip
horitontally ($f$), then we get
\begin{figure*}[h]
  \centering
   \includegraphics[width=6cm]{r_f.pdf}
\end{figure*}
which is just $f_\ell$!  If we do this in the opposite order, doing $f$
then $r$, we get
\begin{figure*}[h]
  \centering
   \includegraphics[width=6cm]{f_r.pdf}
\end{figure*}
then we get something different; it's $f_r$ this time!  So, order matters:
\begin{eqnarray*}
  f \circ r &= f_\ell\\
  r \circ f &= f_r\\
\end{eqnarray*}




\begin{figure*}
   \centering
   \begin{tabular}{c|cccccc}
     & 1 & $r$ & $\ell$ & $f$ & $f_r$ & $f_\ell$ \\
     \hline
     1        & 1        & $r$    & $\ell$    & $f$ & $f_r$ & $f_\ell$ \\
     $r$      & $r$      & $\ell$ & $1$       & $f_\ell$ & $f$   & $f_r$      \\
     $\ell$   & $\ell$   & $1$    & $r$       & $f_r$ & $f_\ell$   & $f$      \\
     $f$      & $f$      & $f_r$  & $f_\ell$   & $1$ & $r$   & $\ell$      \\
     $f_r$    & $f_r$     & $f_\ell$   & $f$    & $\ell$ & $1$   & $r$      \\
     $f_\ell$  & $f_\ell$   & $f$   & $f_r$      & $r$ & $\ell$   & $1$      \\
   \end{tabular}
   \caption{Apply the left option follow by the top option.}
\end{figure*}

\end{document}
