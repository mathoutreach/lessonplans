\documentclass[12pt]{article}

\usepackage{graphicx}
\usepackage{amsmath}
\usepackage{breqn}

\title{Symmetries of equilateral triangles}


\begin{document}

\thispagestyle{empty}

\maketitle

The group of symmetries of an equilateral triangle is size 6.  In
addition to the identity, there are two rotations, and three flips.
We can label the vertices to track how they move under the action of
this group.  While the vertices may move, the shape stays the same; this
is what we mean by symmetries.  They are givne in Figure~\ref{trisym}.
\begin{figure}[h]
   \centering
   \begin{tabular}{|c|c|c|c|}
     \hline Action: & Nothing! & right & left  \\
     \hline
     Symbol: & $1$ & $r$ & $\ell$  \\
     \hline
     Diagram: &
     \includegraphics[width=2cm]{id.pdf} &
     \includegraphics[width=2cm]{r.pdf}  &
     \includegraphics[width=2cm]{l.pdf} \\
     \hline
     \hline Action: & flip & flip right & flip left \\
     \hline
     Symbol: & $f$ & $f_r$ & $f_\ell$ \\
     \hline
     Diagram: &
     \includegraphics[width=2cm]{f.pdf}  &
     \includegraphics[width=2cm]{fr.pdf} &
     \includegraphics[width=2cm]{fl.pdf}\\
     \hline
   \end{tabular}
   \caption{The symmetries of an equilateral triangle.}
  \label{trisym}
\end{figure}

If we were to perform on action and then another action, we would
still get the same shape, but the vertices get moved somewhere else.
Since there are only six possibilities for where they might end up,
each combination of symmetries has got to be one of the six symmetries
that we've already found!

For example, if we first rotate to the right ($r$), and then flip
horitontally ($f$), then we get:
\begin{figure}[h]
  \centering
  \includegraphics[width=6cm]{r_f.pdf}
  \caption{First $r$ then $f$.}
\end{figure}
which is just $f_\ell$!  If we do this in the opposite order, doing $f$
then $r$, we get
\begin{figure}[h]
  \centering
  \includegraphics[width=6cm]{f_r.pdf}
  \caption{First $f$ then $r$.}
\end{figure}
then we get something different; it's $f_r$ this time!  So, order
matters.  If we do $r$ then $f$, we write $f\circ{}r$ and get
$f_\ell$.  If we do $f$ then $r$ we write $r\circ{}f$ and get $f_r$.
That is,
\begin{equation}
  \label{frrf}
  f \circ r = f_\ell \neq r \circ f = f_r.
\end{equation}

So, just like with the $1,\dots,9$, we can make a multiplication table
with these symmetries.  For $1,\dots,9$, the numbers get bigger;
$3\times{}4=12>9$, for example.  But for symmetries, whatever
symmetries we combine, we always end up with one of the 6 symmetries
with which we started - the result is always a triangle that's been
flipped or rotated.  We've figured out two of them in
equation~\eqref{frrf}.  To figure out the other 36, we can just apply
one translation and then another!  However, this is a lot of work;
especially so because it seems easy to make mistakes.  Instead, we can
express the symmetries as permutations of the indices, which is shown
in Figure~\ref{permtab}.
\begin{figure}[h]
   \centering
   \begin{tabular}{|c|c|}
     \hline
     Symmetry & Permutation \\
     \hline \hline
     $1$ & $(1)$ \\
     \hline
     $r$ & $(321)$ \\
     \hline
     $\ell$ & $(231)$ \\
     \hline
     $f$ & $(23)$ \\
     \hline
     $f_r$ & $(12)$ \\
     \hline
     $f_\ell$ & $(13)$ \\
     \hline
   \end{tabular}
   \caption{Symmetries and permutations}
  \label{permtab}
\end{figure}
When we apply two permutations we can use the rules of combining
permutations.  For example,
\begin{dmath}
  r \circ f
  = (321)(23)
  = (12)
  = f_r.
\end{dmath}
This can be a simpler way to figure out the composition of operations.
Performing this on all of the symmetries, get the table in
Figure~\ref{multtab}.
\begin{figure}[h]
  \centering
  \begin{tabular}{c|cccccc|}
    & 1 & $r$ & $\ell$ & $f$ & $f_r$ & $f_\ell$ \\
    \hline
    1        & 1        & $r$    & $\ell$    & $f$ & $f_r$ & $f_\ell$ \\
    $r$      & $r$      & $\ell$ & $1$       & $f_\ell$ & $f$   & $f_r$      \\
    $\ell$   & $\ell$   & $1$    & $r$       & $f_r$ & $f_\ell$   & $f$      \\
    $f$      & $f$      & $f_r$  & $f_\ell$   & $1$ & $r$   & $\ell$      \\
    $f_r$    & $f$     & $f_\ell$   & $f$    & $\ell$ & $1$   & $r$      \\
    $f_\ell$  & $f_\ell$   & $f$   & $f_r$      & $r$ & $\ell$   & $1$      \\
    \hline
  \end{tabular}
  \caption{Multiplication table for the symmetries of the triangle:
    apply the left option follow by the top option.}
  \label{multtab}
\end{figure}

\end{document}
