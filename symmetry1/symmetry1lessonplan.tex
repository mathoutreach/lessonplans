\documentclass[12pt]{article}

\usepackage{graphicx}
\usepackage{amsmath}
\usepackage{breqn}

\title{Symmetries 1 Lesson Plan}

\begin{document}

\thispagestyle{empty}

\maketitle

This lesson plan is for a 1 / 1.5 hour class of early elementary
students.  The theme of this lesson is symmetries of the equilateral
triangle.  An instructor handout is available in the
\texttt{handouts/trisym} directory.  Everything that I have ever done
has taken more time than I expected, so it's quite likely that this
lesson plan is overly optimistic, and we'll need to cut.  The last
activity, snowflake cutting, can be removed or left as a take-home
activity.

Lesson plan:
\begin{enumerate}
\item{Introduce symmetries of the triangle [20 minutes]}\\ The
  equilateral triangle has 6 symmetries, including the identity, which
  forms a group.  Our objective here is to
  \begin{itemize}
  \item Identify the symmetries
  \item Compose symmetries (eg a rotation + flip = another flip).
  \item Make the link between addition as an operation between
    numbers, and composition as an operation between symmetries.
  \item Notice that the symmetries do not commute, unlike addition of
    integers.
  \end{itemize}
  Talk about the symmetries, and see if we can, as class, figure out
  the composition of a rotation and a flip, and notice that it's
  different than a flip and a rotation.  We can also allude to the
  multiplication table of the group of symmetries, though filling it
  in might be too much. (But we could leave that as an exercise that
  the students can bring home for the parents!)
  
  We can make a connection between the symmetries of the equilateral
  triangle and that of other solids.  For example, the square has 8
  symmetries, the cube has 48, and the Rubik's Cube group is about
  $4.3252 \times 10^{19}$
\item{Tri-nim game [20 minutes]}\\ Tri-nim (see \texttt{trinim.pdf} in
  this directory) is a modification of the classic John-Nash
  game-theory game Nim.  The winning strategy for this game involves
  symmetries; students are encouraged to play for a while and see how
  things go.  The usual way to roll out this game is to give a
  demonstration, and then let students play in groups while they try
  to figure out a good strategy.

  Once they've played a while, ask the students if anyone has a good
  strategy, and invite them to play against the teacher and/or the
  outreach instructor.  Ask the class to try and explain the strategy.
  Links can be made to tic-tac-toe.

  Questions and directions of inquiry:
  \begin{itemize}
  \item Is there an advantage for the first or second player?
  \item Can we tie?
  \item What kind of strategy works?
  \item Does the game change if there are a different number of spots
    on the sides of the triangle?
  \end{itemize}
\item{Snowflakes with hexagonal symmetry [20+ minutes]}\\ Water
  molecules are shaped like tetrahedra; these combine together into
  hexagons.  That's why snowflakes have hexagonal symmetry; it's the shape
  of the water molecules!

  The activity in this section is to distinguish between incorrect
  8-sided paper snowflakes and 6-sided snowflakes.  First, demonstrate
  how folding a square piece of paper can generate an 8-sided
  snowflake.  To get the hexagonal snowflake, fold in half, half
  again, and then in thirds.
\end{enumerate}


\end{document}

% LocalWords:  trisym eg Tri nim trinim pdf tac tetrahedra
